% Options for packages loaded elsewhere
\PassOptionsToPackage{unicode}{hyperref}
\PassOptionsToPackage{hyphens}{url}
%
\documentclass[
  12pt,
  a4paper]{article}
\usepackage{amsmath,amssymb}
\usepackage{iftex}
\ifPDFTeX
  \usepackage[T1]{fontenc}
  \usepackage[utf8]{inputenc}
  \usepackage{textcomp} % provide euro and other symbols
\else % if luatex or xetex
  \usepackage{unicode-math} % this also loads fontspec
  \defaultfontfeatures{Scale=MatchLowercase}
  \defaultfontfeatures[\rmfamily]{Ligatures=TeX,Scale=1}
\fi
\usepackage{lmodern}
\ifPDFTeX\else
  % xetex/luatex font selection
\fi
% Use upquote if available, for straight quotes in verbatim environments
\IfFileExists{upquote.sty}{\usepackage{upquote}}{}
\IfFileExists{microtype.sty}{% use microtype if available
  \usepackage[]{microtype}
  \UseMicrotypeSet[protrusion]{basicmath} % disable protrusion for tt fonts
}{}
\makeatletter
\@ifundefined{KOMAClassName}{% if non-KOMA class
  \IfFileExists{parskip.sty}{%
    \usepackage{parskip}
  }{% else
    \setlength{\parindent}{0pt}
    \setlength{\parskip}{6pt plus 2pt minus 1pt}}
}{% if KOMA class
  \KOMAoptions{parskip=half}}
\makeatother
\usepackage{xcolor}
\usepackage[margin=1in]{geometry}
\usepackage{color}
\usepackage{fancyvrb}
\newcommand{\VerbBar}{|}
\newcommand{\VERB}{\Verb[commandchars=\\\{\}]}
\DefineVerbatimEnvironment{Highlighting}{Verbatim}{commandchars=\\\{\}}
% Add ',fontsize=\small' for more characters per line
\usepackage{framed}
\definecolor{shadecolor}{RGB}{248,248,248}
\newenvironment{Shaded}{\begin{snugshade}}{\end{snugshade}}
\newcommand{\AlertTok}[1]{\textcolor[rgb]{0.94,0.16,0.16}{#1}}
\newcommand{\AnnotationTok}[1]{\textcolor[rgb]{0.56,0.35,0.01}{\textbf{\textit{#1}}}}
\newcommand{\AttributeTok}[1]{\textcolor[rgb]{0.13,0.29,0.53}{#1}}
\newcommand{\BaseNTok}[1]{\textcolor[rgb]{0.00,0.00,0.81}{#1}}
\newcommand{\BuiltInTok}[1]{#1}
\newcommand{\CharTok}[1]{\textcolor[rgb]{0.31,0.60,0.02}{#1}}
\newcommand{\CommentTok}[1]{\textcolor[rgb]{0.56,0.35,0.01}{\textit{#1}}}
\newcommand{\CommentVarTok}[1]{\textcolor[rgb]{0.56,0.35,0.01}{\textbf{\textit{#1}}}}
\newcommand{\ConstantTok}[1]{\textcolor[rgb]{0.56,0.35,0.01}{#1}}
\newcommand{\ControlFlowTok}[1]{\textcolor[rgb]{0.13,0.29,0.53}{\textbf{#1}}}
\newcommand{\DataTypeTok}[1]{\textcolor[rgb]{0.13,0.29,0.53}{#1}}
\newcommand{\DecValTok}[1]{\textcolor[rgb]{0.00,0.00,0.81}{#1}}
\newcommand{\DocumentationTok}[1]{\textcolor[rgb]{0.56,0.35,0.01}{\textbf{\textit{#1}}}}
\newcommand{\ErrorTok}[1]{\textcolor[rgb]{0.64,0.00,0.00}{\textbf{#1}}}
\newcommand{\ExtensionTok}[1]{#1}
\newcommand{\FloatTok}[1]{\textcolor[rgb]{0.00,0.00,0.81}{#1}}
\newcommand{\FunctionTok}[1]{\textcolor[rgb]{0.13,0.29,0.53}{\textbf{#1}}}
\newcommand{\ImportTok}[1]{#1}
\newcommand{\InformationTok}[1]{\textcolor[rgb]{0.56,0.35,0.01}{\textbf{\textit{#1}}}}
\newcommand{\KeywordTok}[1]{\textcolor[rgb]{0.13,0.29,0.53}{\textbf{#1}}}
\newcommand{\NormalTok}[1]{#1}
\newcommand{\OperatorTok}[1]{\textcolor[rgb]{0.81,0.36,0.00}{\textbf{#1}}}
\newcommand{\OtherTok}[1]{\textcolor[rgb]{0.56,0.35,0.01}{#1}}
\newcommand{\PreprocessorTok}[1]{\textcolor[rgb]{0.56,0.35,0.01}{\textit{#1}}}
\newcommand{\RegionMarkerTok}[1]{#1}
\newcommand{\SpecialCharTok}[1]{\textcolor[rgb]{0.81,0.36,0.00}{\textbf{#1}}}
\newcommand{\SpecialStringTok}[1]{\textcolor[rgb]{0.31,0.60,0.02}{#1}}
\newcommand{\StringTok}[1]{\textcolor[rgb]{0.31,0.60,0.02}{#1}}
\newcommand{\VariableTok}[1]{\textcolor[rgb]{0.00,0.00,0.00}{#1}}
\newcommand{\VerbatimStringTok}[1]{\textcolor[rgb]{0.31,0.60,0.02}{#1}}
\newcommand{\WarningTok}[1]{\textcolor[rgb]{0.56,0.35,0.01}{\textbf{\textit{#1}}}}
\usepackage{longtable,booktabs,array}
\usepackage{calc} % for calculating minipage widths
% Correct order of tables after \paragraph or \subparagraph
\usepackage{etoolbox}
\makeatletter
\patchcmd\longtable{\par}{\if@noskipsec\mbox{}\fi\par}{}{}
\makeatother
% Allow footnotes in longtable head/foot
\IfFileExists{footnotehyper.sty}{\usepackage{footnotehyper}}{\usepackage{footnote}}
\makesavenoteenv{longtable}
\usepackage{graphicx}
\makeatletter
\def\maxwidth{\ifdim\Gin@nat@width>\linewidth\linewidth\else\Gin@nat@width\fi}
\def\maxheight{\ifdim\Gin@nat@height>\textheight\textheight\else\Gin@nat@height\fi}
\makeatother
% Scale images if necessary, so that they will not overflow the page
% margins by default, and it is still possible to overwrite the defaults
% using explicit options in \includegraphics[width, height, ...]{}
\setkeys{Gin}{width=\maxwidth,height=\maxheight,keepaspectratio}
% Set default figure placement to htbp
\makeatletter
\def\fps@figure{htbp}
\makeatother
\setlength{\emergencystretch}{3em} % prevent overfull lines
\providecommand{\tightlist}{%
  \setlength{\itemsep}{0pt}\setlength{\parskip}{0pt}}
\setcounter{secnumdepth}{-\maxdimen} % remove section numbering
\ifLuaTeX
  \usepackage{selnolig}  % disable illegal ligatures
\fi
\IfFileExists{bookmark.sty}{\usepackage{bookmark}}{\usepackage{hyperref}}
\IfFileExists{xurl.sty}{\usepackage{xurl}}{} % add URL line breaks if available
\urlstyle{same}
\hypersetup{
  hidelinks,
  pdfcreator={LaTeX via pandoc}}

\author{}
\date{\vspace{-2.5em}}

\begin{document}

\begin{longtable}[]{@{}lrrr@{}}
\toprule\noalign{}
AE & SME & Shapley & HN \\
\midrule\noalign{}
\endhead
\bottomrule\noalign{}
\endlastfoot
\textless{} 0.6 & 0.7753 & 0.7722 & 0.759 \\
\textgreater= 0.6 & 0.5282 & 0.5723 & 0.561 \\
\end{longtable}

To analyse which attribution method provides the most accurate
explanation, the Spearman rank correlation is computed between the
substructures attributions and a chemically intuitive ranking of those
substructures. Also the influence of the absolute error between model
prediction and experimental value will be examined by subdividing the
molecules into two \((<0.6\) or \(>=0.6)\). Since the distributions of
the Spearman rank correlations are highly skewed to the left,
non-parametric test are used for the analysis. To test whether there is
a significant difference of rank correlation between the attribution
methods two Friedman test will be used (one for each absolute error
category). For testing the presence of a significant difference between
the AE groups a three Wilcoxon--Mann--Whitney test will be applied.
Friedman tests will also be used in the analysis of fidelity pos and
fidelity negative over the attribution methods. Control for the multiple
test will be done using Bonferonni, resulting in a significance level of
\(0.05/5 = 0.01\).

\includegraphics{statistical_test_files/figure-latex/distributions-1.png}

\newpage

\hypertarget{friedman-test-of-attribution-methods}{%
\subsubsection{Friedman test of attribution
methods}\label{friedman-test-of-attribution-methods}}

The presence of a statistical difference in Spearman rank correlation
between attributions and chemically intuitive ranks for different
attribution methods is tested using a Friedman test. The null-hypothesis
assumes that there is no difference between the different attribution
methods, which would result in similar rank sums of the different
attribution methods. For both groups of absolute error, smaller than
\(0.6\) (p-value 0.217) and larger than or equal to \(0.6\) (p-value
0.152), are not significant based on the bonferonni corrected nominal
level of \(0.01\). Therefore, it is concluded that there is not enough
evidence in the data to reject the null hypothesis and it is assumed the
the method of computing the attribution values has not a significant
impact in the accurateness of the resulting explanation.

Table of Friedman test result for molecules with absolute prediction
error smaller than \(0.6\):

\begin{longtable}[]{@{}lrrrrl@{}}
\toprule\noalign{}
.y. & n & statistic & df & p & method \\
\midrule\noalign{}
\endhead
\bottomrule\noalign{}
\endlastfoot
rank\_corr & 72 & 3.058823 & 2 & 0.2166631 & Friedman test \\
\end{longtable}

Table of Friedman test result for molecules with absolute prediction
error larger than or equal to \(0.6\):

\begin{longtable}[]{@{}lrrrrl@{}}
\toprule\noalign{}
.y. & n & statistic & df & p & method \\
\midrule\noalign{}
\endhead
\bottomrule\noalign{}
\endlastfoot
rank\_corr & 28 & 3.769231 & 2 & 0.1518875 & Friedman test \\
\end{longtable}

\hypertarget{wilcoxon-mann-whitney-test-between-absolute-error-class}{%
\subsubsection{Wilcoxon-Mann-Whitney test between absolute error
class}\label{wilcoxon-mann-whitney-test-between-absolute-error-class}}

\begin{verbatim}
## 
##  Wilcoxon rank sum test with continuity correction
## 
## data:  rank_corr by absolute_error_class
## W = 1254.5, p-value = 0.03526
## alternative hypothesis: true location shift is not equal to 0
\end{verbatim}

\begin{verbatim}
## 
##  Wilcoxon rank sum test with continuity correction
## 
## data:  rank_corr by absolute_error_class
## W = 1132, p-value = 0.2807
## alternative hypothesis: true location shift is not equal to 0
\end{verbatim}

\begin{verbatim}
## 
##  Wilcoxon rank sum test with continuity correction
## 
## data:  rank_corr by absolute_error_class
## W = 1177.5, p-value = 0.1455
## alternative hypothesis: true location shift is not equal to 0
\end{verbatim}

In all attribution methods there is no significance difference between
the two different absolute error groups (p-values are 0.0352554,
0.2807212, 0.1454535 for SME, Shapley and HN respectively ). Therefore,
it is concluded that there is not enough evidence in the data to reject
the null hypothesis of equal distributions.

\hypertarget{power-calculations}{%
\subsection{Power calculations}\label{power-calculations}}

\begin{Shaded}
\begin{Highlighting}[]
\CommentTok{\# Verify location shift}
\NormalTok{rank\_corr\_SME }\OtherTok{\textless{}{-}}\NormalTok{ rank\_corr\_long }\SpecialCharTok{\%\textgreater{}\%} \FunctionTok{filter}\NormalTok{(attribution\_method }\SpecialCharTok{==} \StringTok{"SME\_rank\_corr"}\NormalTok{)}
\NormalTok{X }\OtherTok{\textless{}{-}} \FunctionTok{as.list}\NormalTok{(rank\_corr\_SME }\SpecialCharTok{\%\textgreater{}\%} \FunctionTok{filter}\NormalTok{(absolute\_error\_class }\SpecialCharTok{==} \StringTok{"\textless{} 0.6"}\NormalTok{) }\SpecialCharTok{\%\textgreater{}\%} \FunctionTok{select}\NormalTok{(rank\_corr))}\SpecialCharTok{$}\NormalTok{rank\_corr}
\NormalTok{Y }\OtherTok{\textless{}{-}} \FunctionTok{as.list}\NormalTok{(rank\_corr\_SME }\SpecialCharTok{\%\textgreater{}\%} \FunctionTok{filter}\NormalTok{(absolute\_error\_class }\SpecialCharTok{==} \StringTok{"\textgreater{}= 0.6"}\NormalTok{) }\SpecialCharTok{\%\textgreater{}\%} \FunctionTok{select}\NormalTok{(rank\_corr))}\SpecialCharTok{$}\NormalTok{rank\_corr}

\FunctionTok{qqplot}\NormalTok{(X, Y)}
\end{Highlighting}
\end{Shaded}

\includegraphics{statistical_test_files/figure-latex/unnamed-chunk-6-1.pdf}

\begin{Shaded}
\begin{Highlighting}[]
\NormalTok{N\_sim }\OtherTok{\textless{}{-}} \DecValTok{10000}
\NormalTok{N\_x }\OtherTok{\textless{}{-}} \DecValTok{10}
\NormalTok{N\_y }\OtherTok{\textless{}{-}} \DecValTok{10}
\NormalTok{p\_values }\OtherTok{\textless{}{-}} \FunctionTok{rep}\NormalTok{(}\ConstantTok{NA}\NormalTok{, N\_sim)}

\ControlFlowTok{for}\NormalTok{ (i }\ControlFlowTok{in} \DecValTok{1}\SpecialCharTok{:}\NormalTok{N\_sim) \{}
\NormalTok{  X }\OtherTok{\textless{}{-}} \FunctionTok{rnorm}\NormalTok{(N\_x, }\AttributeTok{mean =} \DecValTok{0}\NormalTok{, }\AttributeTok{sd =} \DecValTok{1}\NormalTok{)}
\NormalTok{  Y }\OtherTok{\textless{}{-}} \FunctionTok{rnorm}\NormalTok{(N\_y, }\AttributeTok{mean =} \DecValTok{1}\NormalTok{, }\AttributeTok{sd =} \DecValTok{1}\NormalTok{)}
  
\NormalTok{  p\_values[i] }\OtherTok{\textless{}{-}} \FunctionTok{wilcox.test}\NormalTok{(X, Y)}\SpecialCharTok{$}\NormalTok{p.value}
\NormalTok{\}}

\FunctionTok{mean}\NormalTok{(p\_values }\SpecialCharTok{\textless{}=} \FloatTok{0.05} \SpecialCharTok{/} \DecValTok{5}\NormalTok{)}
\end{Highlighting}
\end{Shaded}

\begin{verbatim}
## [1] 0.2623
\end{verbatim}

\end{document}
