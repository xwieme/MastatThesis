\chapter{Hamiache-Navarro value example}
\label{app:HN_example}

For the unanimity $(\{1, 2, 3\}, u_{\{1, 2, 3\}}, \{\{1, 2\}, \{2, 3\}\})$, the matrices $M_c$ and $P_g$ are 
given by\cite{hamiache_associated_2020}


\begin{equation}
	M_c = \begin{pmatrix}
		1 - 2\tau & -\tau     & -\tau     & \tau     & \tau     & 0        & 0    \\
		-\tau     & 1 - 2\tau & -\tau     & \tau     & 0        & \tau     & 0    \\
		-\tau     & -\tau     & 1 - 2\tau & 0i       & \tau     & \tau     & 0    \\
		0         & 0         & -\tau     & 1 - \tau & 0        & 0        & \tau \\
		0         & -\tau     & 0         & 0        & 1 - \tau & 0        & \tau \\
		-\tau     & 0         & 0         & 0        & 0        & 1 - \tau & \tau \\
		0         & 0         & 0         & 0        & 0        & 0        & 1    \\
	\end{pmatrix},
\end{equation}


\begin{equation}
	\renewcommand{\arraystretch}{0.7}
	P_g = \begin{pmatrix}
		1 & 0 & 0 & 0 & 0 & 0 & 0 \\
		0 & 1 & 0 & 0 & 0 & 0 & 0 \\
		0 & 0 & 1 & 0 & 0 & 0 & 0 \\
		0 & 0 & 0 & 1 & 0 & 0 & 0 \\
		1 & 0 & 1 & 0 & 0 & 0 & 0 \\
		0 & 0 & 0 & 0 & 0 & 1 & 0 \\
		0 & 0 & 0 & 0 & 0 & 0 & 1 \\
	\end{pmatrix}.
\end{equation}


The vector representation of the characteristic function is $u_{\{1, 2, 3\}} = \left(\begin{smallmatrix} 0 & 0 & 0 & 0 & 0 & 0 & 1 \end{smallmatrix}\right)^T$.
Then the limit game $\tilde{v} = (P_g M_c P_g)^k u_{\{1, 2, 3\}}$ equals 
$\left(\begin{smallmatrix} 1/4 & 1/2 & 1/4 & 3/4 & 1/2 & 3/4 & 1\end{smallmatrix}\right)^T$. Since the solution  of the 
original game equals the payoff of the limit game, the values of players one, two and three are $1/4, 1/2$ and $1/4$ respectively.


\chapter{Functional groups}
\label{app:functional_groups_list}


\begin{table}[ht]
   \centering
   \caption{SMARTS of functional groups used to partition a molecule}
   \begin{tabular}{cc}
        \toprule
        \textbf{Functional group} & \textbf{SMARTS} \\
       \midrule
    R-tBu & [C,c]-C([C;H3])([C;H3])[C;H3]\\
    R-CF3 & [C,c]-C(F)(F)F\\
    R-N=NH & [C,c]-N=[N;H]\\
    R-N=C=S & [C,c]-N=C=S\\
    R-N=C=O & [C,c]-N=C=O\\
    R-CN & [C,c]-C\#N\\
    R-C=NH & [C,c]-C=[N;H]\\
    R-C=N-CH3 & [C,c]-C=N-[C;H3]\\
    R-C=N-OH & [C,c]-C=N-[O;H]\\
    R-C\#CH & [C,c]-C\#[C;H]\\
    R-NO2 & [C,c]-N(=O)O\\
    R-NHSO2CH3 & [C,c]-[N;H]-S(=O)(=O)-[C;H3]\\
    R-SO2NH2 & [C,c]-S(=O)(=O)[N;H2]\\
    R-SO2CH3 & [C,c]-S(=O)(=O)[C;H3]\\
    R-SO2Cl & [C,c]-S(=O)(=O)Cl\\
    R-SO2OCH3 & [C,c]-OS(=O)(=O)[C;H3]\\
    R-SO3H & [C,c]-S(=O)(=O)[O;H]\\
    R-S(=O)CH3 & [C,c]-S(=O)[C;H3]\\
    R-C(=O)OCH3 & [C,c]-C(=O)O[C;H3]\\
    R-C(=O)OEt & [C,c]-C(=O)O[C;H2][C;H3]\\
    R-OEt & [C,c]-O[C;H2][C;H3]\\
    R-OMe & [C,c]-O[C;H3]\\
    R-SCH3 & [C,c]-S[C;H3]\\
    R-N-C(=O)CH3 & [C,c]-N-C(=O)[C;H3]\\
    R-C(=O)NH2 & [C,c]-C(=O)[N;H2]\\
    R-C(=O)OH & [C,c]-C(=O)[C;H3]\\
    R-C(=O)CH3 & [C,c]-C(=O)[C;H3]\\
    R-N=O & [C,c]-N=O\\
    R=O & C(=O)\\
    R=S & C(=S)\\
    R-NH2 & [C,c]-[N;H2]\\
    RX & [C,c]-[F,Cl,Br,I]\\
    ROH & [C,c]-[O;H]\\
    RSH & [C,c]-[S;H]\\
%    Phenyl & [C,N]-c1[c;h1][c;h1][c;h1][c;h1][c;h1]1\\
\bottomrule
\end{tabular}
\end{table}



