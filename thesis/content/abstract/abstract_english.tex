\chapter{Abstract}


Prediction of molecular properties is a crucial task in Compute-Aided 
Drug discovery (CADD), since it allows for the screening of substantial 
data sets to select only promising compounds. Therefore, fewer wet lab experiments are necessary, 
resulting in a reduction of both time and cost of the drug discovery process.\cite{adelusi2022molecular} 
Due to the availability of experimental data sets, machine learning 
algorithms (ML) are commonly used to predict molecular properties.\cite{wu2018moleculenet, gorgulla2022emerging, tropsha2024integrating} 
However, the black-box nature of ML algorithms 
results in a limitation on the creation of novel scientific knowledge. Also, 
the opaqueness of ML models could lead to trust issues, especially 
in performance-critical and highly regulated environments.\cite{carvalho2019machine}


% Prediction of molecular properties allows the filtering of molecular data sets, 
% resulting in fewer wet lab experiments and a reduced development time of novel 
% drugs. Quantitative Structure-Activity Relationships (QSAR)
% correlate the molecular structure with physical properties or biological activity. 
% \cite{} QSAR models can then predict molecular properties 
% based on their structural information of various molecular structures 
% can then filter out promising molecules using QSAR models.
% 
% The growing availability of experimental data sets allows the utilization of machine learning (ML)
% in QSAR modeling.  
% However, ML algorithms are black-boxes, which limits the creation of novel scientific 
% knowledge and can result in trust issues in performance-critical and highly regulated 
% environments.


Explainable artificial intelligence (XAI) tries to get insight into black-box 
ML models.\cite{carvalho2019machine, yuan2022explainability} 
Current XAI methods focus on the ML model and often neglect domain 
knowledge of the studied problem. For example, when molecules are represented 
as graphs using the Lewis structure\cite{ahmad1992drawing}, common XAI techniques will search for the 
most important nodes or subgraphs to explain molecular property prediction.\cite{wu2023chemistry} The 
resulting explanation, however, is not chemically relevant as chemists are used 
to thinking in terms of substructures such as functional groups. Recently, Wu et. 
al. developed an attribution based XAI method that first partitions the molecular graph into 
chemically meaningful substructures. Therefore, their method, substructure graph 
exploration (SME), can produce chemically intuitive explanations.\cite{wu2023chemistry}


SME takes the difference between the model prediction of the molecular graph and 
the prediction of the molecular graph in which a substructure is masked. Yet, 
this approach does not consider the interaction between molecular substructures.
A commonly used attribution method in machine learning that does consider 
interactions between features is the Shapley value. Here, the Shapley value\cite{shapley1953value} will 
be used to compute the attributions of molecular graphs using a similar approach 
of Wu et. al. Since the Shapley value does not take the graphical structure of 
the input into account, the Hamiache-Navarro value\cite{hamiache_value_1999} 
is also considered to investigate whether the inclusion of the graphical structure 
can significantly improve the explanation. Comparison of the different attribution 
methods is done using pairwise Spearman rank correlation between attributions of 
different methods. The fidelity\cite{carvalho2019machine} is used to measure 
how faithful the explanation is to the model. Spearman rank correlation is also 
used to compare the ranked attributions to a manual ranks obtained using chemical 
intuition. 


The explanation methods are applied to a machine learning model that predicts the 
estimated water solubility in logS (with S the solubility in $mol/L$). Water 
solubility is chosen as the relative water solubility of a compound can be estimated 
from its molecular structure. Also, the prediction of water solubility is used 
in drugs discovery as this influences the drug blood concentration.\cite{hill2010getting}
