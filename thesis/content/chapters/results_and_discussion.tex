\chapter{Results and discussion}

\section{RGCN is able to accurately predict expected water solubility}


The resulting ten RGCN models, each with a different seed, have comparable results 
(\cref{fig:training_history}). 
Performance of the training data is relatively good, yielding an average $R^2$ 
of $0.9630$ and an average mean squared error (MSE) of $0.0744$ (\cref{tab:model_performance}). 
The model has a similar $R^2$ for the validation data, however, larger errors are made. 
Model performance on the test data is similar as the validation performance, showing that the 
model is able to generalize well to unseen data. Considering that most experimental data have an RMSE 
between $0.6$ and $0.7$ log(mol/L),\cite{palmer2014experimental} 
it can be concluded that the resulting model can predict water solubility of small 
molecules with decent performance.

% \begin{figure}[h]
%     \centering
%     \includegraphics[scale=0.20]{rgcn_r2.png}
%     \includegraphics[scale=0.20]{rgcn_mse.png}
%     \caption{Average Mean squared error (left) and $R^2$ (right) of train (orange) and validation 
%     (blue) data during model training over the ten RGCN models, each trained using a different 
%     seed. Both metrics show a good model performance and no overfitting.}
% \end{figure}


\begin{figure}[h]
    \centering
    \includegraphics[scale=0.20]{rgcn_r2_zoomed.png}
    \includegraphics[scale=0.20]{rgcn_mse_zoomed.png}
    \caption{Average $R^2$ (left) and mean squared error (right) of train (orange) and validation 
    (blue) data during model training over the ten RGCN models, each trained using a different 
    seed. Both metrics show a good model performance and no overfitting. }
    \label{fig:training_history}
\end{figure}


\begin{table}[h]
    \centering
    \caption{Average $R^2$ and MSE for train, validation and test data sets with standard deviation between brackets.}
    \label{tab:model_performance}
    \begin{tabular}{ccc}
        \toprule 
       & $\pmb{R^2}$ & \textbf{MSE $\pmb{\left(\left[log(mol/L)\right]^2\right)}$} \\
        \midrule
        \textbf{Train} & $0.9630 (\pm 0.0001)$ & $0.0744 (\pm 0.0031)$ \\
        \textbf{Validation} & $0.9225 (\pm 0.0076)$ & $0.3354 (\pm 0.0120)$ \\
        \textbf{Test} & $0.9079 (\pm 0.0050)$ & $0.3193 (\pm 0.0047)$ \\
        \bottomrule 
    \end{tabular}
\end{table}



Currently, it is questionable whether the ML model effectively learns chemistry and 
if wrong predictions may be explained by chemically incorrect reasoning of the model. 
To address this, different attribution methods (SME, Shapley value and HN value) are 
compared to each other and with chemical theory. It should be noted that these attribution 
methods can reveal insights into the chemical reasoning of the model, but are not necessarily 
causal. This means that if the attributions show inconsistency with chemical theory, other 
factors could also influence the wrong prediction.


\section{Different attribution methods can result in different explanations}


The attribution values sign usually shows an agreement with expectations from 
chemistry (\cref{fig:attribution_distribution_fg}). Groups that positively affect the polarity of a molecule have a 
positive attribution, while unfavorable functional groups for water solubility 
have a negative attribution. The median of HN values also shows a correct 
relationship between the functional groups. Hydroxyl groups are hydrogen bond 
acceptors and donors, so it has a superior influence on water solubility than 
a methyl ester that can only accept hydrogen bonds. Also, ethoxy has a lower 
HN value than methoxy, which is due to the larger carbon chain of ethoxy.


\begin{figure}[h]
    \centering
    \includegraphics[scale=0.5]{attribution_distribution_functional_groups.png}
    \caption{The Shapley value, HN value, and SME attribution distributions for selected functional groups.}
    \label{fig:attribution_distribution_fg}
\end{figure}


Shapley values have mostly a broader IQR relative to the HN values. The median 
Shapley value of alcohols is lower than methyl esters, which is unexpected. 
However, SME assigns negative attributions to hydroxyl groups, which incorrectly suggests 
that hydroxyl groups can decrease the solubility of small molecules. One of those 
molecules is erythritol (\cref{fig:erythritol_explanation}), which has a predicted solubility of 0.432 log(mol/L) 
and an experimental solubility of 0.700 log(mol/L). Since the absolute prediction 
error is within acceptable limits, it would be possible that the model achieved 
this in a chemically incorrect way (i.e. clever Hans effect\cite{lapuschkin2019unmasking}). 
Clever Hans effect is a correct model prediction, however, due to the wrong reason. 
The clever Hans effect is supported by the wrong Shapley value explanation as it gives the 
apolar carbon chain a higher attribution than the hydroxyl groups. However, 
the explanation of the HN values is chemically correct, which makes it difficult 
to judge whether there is a clever Hans effect. 

Furthermore, the absolute prediction 
error for only five of the sixteen molecules with a negative SME attribution for 
hydroxyl groups is greater than the experimental error. However, those five molecules 
have a chemically consistent explanation from Shapley and HN values. The fact that only 
five of the wrong SME explanations are indeed wrongly predicted and the other attribution 
are in agreement with chemical theory tends to show that SME does not provide the correct 
model explanation. To obtain better understanding of the agreement/disagreement between 
attribution methods and the relation with absolute prediction error, a ranked based analysis is performed. 


\begin{figure}[h]
    \centering
    \includegraphics[scale=0.85]{erythritol_explanations.png}
    \caption{Explanations of erythritol: SME is wrong due to the negative attributions on the hydroxyl groups, 
    Shapley values are wrong since hydroxyl groups have a lower attribution than the carbon chain, HN values provide 
    a chemically correct explanation. Predicted solubility is 0.432 log(mol/L) and the experimental water solubility is 
    0.700 log(mol/L).}
    \label{fig:erythritol_explanation}
\end{figure}


\section{Absolute prediction error is not associated with the Spearman rank correlation between different attribution methods}


To get more insight into how much the different attribution method agree/disagree 
with each other and whether disagreement can be explained by the absolute prediction 
error, a ranked based approach is used. All substructures are ranked from low attributions 
(i.e. most hydrophobic) towards high attributions (i.e. most hydrophylic). Subsequently,
the Pearson correlation is computed between the ranks of two attribution methods. A 
a negative association between prediction error and Spearman rank correlation
between two attribution methods would show that one of the attribution methods is not 
able to identify these large prediction errors. 


SME and Shapley value attribution rankings agree most often with each other, $78.07\%$
of the molecules have a Spearman rank correlation of one. Two molecules have 
a complete opposite explanation, where the correct ranking is provided by the 
Shapley value based ranking. Overall, no association is observed between Spearman 
correlation and the absolute prediction error. 

Perfect agreement between HN value and Shapley value ranks happens for 
$74.26\%$ of the molecules. Again at low correlation $( \le -0.5)$, the ranking based on 
the Shapley value is most often correct. Also no association between prediction error 
is observed, showing that the addition of graphical structure in the explanation 
method does not systematically change the explanation.


 

\begin{figure}[h]
    \centering
    % \includegraphics[scale=0.35]{../data/images/esol_rank_vs_AE_SME_Shapley_combined.png}
    % \includegraphics[scale=0.35]{../data/images/esol_rank_vs_AE_Shapley_HN_combined.png}
    \includegraphics[scale=0.75]{rmse_vs_rank_corr_attributions.png}
    \caption{Absolute error of model prediction in function of Spearman rank correlation between 
        attributions from SME and Shapley (left) and between Shapley and HN (right) using the full 
        data set containing 1110 molecules. The red line shows experimental error of 0.6 log(mol/L).
    }
\end{figure}


\section{All attribution methods have a similar ranking with chemical expectations}

% \section{Relative evaluation of attribution methods using chemically intuitively ranked substructures shows no statistical
% significant difference between the attribution methods}


To analyze which attribution method provides the most accurate
explanation, the Spearman rank correlation is computed between the
substructures attributions and a chemically intuitive ranking of those
substructures for $100$ molecule from the test data set. Also the influence of the absolute error between model
prediction and experimental value will be examined by subdividing the
molecules into two error groups based on experimental error $(<0.6 \text{ or } >=0.6)$. Since the distributions of
the Spearman rank correlations are highly skewed to the left,
non-parametric tests are used for the analysis (\cref{fig:spearman_corr_manual}). To test whether there is
a significant difference of rank correlation between the attribution
methods, two Friedman tests will be used (one for each absolute error
category). For testing the presence of a significant difference between
the absolute error (AE) groups three Wilcoxon--Mann--Whitney tests will be applied.
Control for the multiple test will be done using Bonferonni correction, 
resulting in a significance level of $0.05/5 = 0.01$.


\begin{figure}[h]
    \centering
    \includegraphics[scale=0.5]{spearman_rank_correlation_manual_vs_attribution.png}
    \caption{Distribution of Spearman rank correlation between an attribution method 
        and a manual ranking of substructures using chemical reasoning. The difference 
        between the two absolute error groups is mainly the intensity of high 
        correlation values.
    }
    \label{fig:spearman_corr_manual}
\end{figure}


\subsection{Friedman analysis shows no statistical significant difference in relative accuracy between the attribution methods}


The presence of a statistical difference in Spearman rank correlation
between attributions and chemically intuitive ranks for different
attribution methods is tested using a Friedman test. The null-hypothesis
assumes that there is no difference between the different attribution
methods, which would result in similar rank sums of the different
attribution methods. For both groups of absolute error, smaller than
$(0.6)$ (p-value $0.0154$) and larger than or equal to $(0.6)$ (p-value
$0.0148$), are not significant based on the bonferonni corrected nominal
level of $(0.01)$ (\cref{fig:friedman_results}). Therefore, it is concluded that there is not enough
evidence in the data to reject the null hypothesis and it is assumed the
the method of computing the attribution values has not a significant
impact in the accurateness of the resulting explanation. However, considering 
that the current sample size equals $100$ and the p-values are close to the 
nominal level, it is recommended to confirm these findings in a follow up 
study with a larger sample size.


\begin{table}[h]
    \centering
    \caption{Average Spearman rank correlation between chemically ranked substructures and 
    a ranking based on attribution values.}
    \begin{tabular}{cccc}
    \toprule
    AE & SME & Shapley & HN \\
    \midrule
     < 0.6 & 0.7767 & 0.7708 & 0.6891 \\
    $\ge$ 0.6 & 0.5136 & 0.5482 & 0.3732 \\
     \bottomrule
    \end{tabular}
\end{table}


\begin{table}[h]
    \centering
    \caption{Friedman test results}
    \label{fig:friedman_results}
    \begin{tabular}{ccccc}
    \toprule
    AE & N & Statistic & df & p \\
    \midrule
     < 0.6 & 72 & 8.348 & 2 & 0.0154  \\
    $\ge$ 0.6 & 28 & 8.432 & 2 & 0.0148 \\
     \bottomrule
    \end{tabular}
\end{table}


\subsection{Molecules with larger prediction error do not have lower relative accuracy}


In all attribution methods there is no significance difference between the two different
absolute error groups (p-values are $0.0214$, $0.1694$, $0.0789$ for SME, Shapley and
HN respectively). Therefore, it is concluded that there is not enough evidence in the
data to reject the null hypothesis of equal distributions. In other words, no attribution 
method has a higher disagreement with chemical expectations when the prediction error 
is above experimental error.


\section{All attribution methods are equally faithful to the ML model}


So far, different attribution methods have been compared in a more qualitative way. 
This is due to the difficulty of quantifying how correct an explanation is. The 
Spearman rank correlation between attribution and chemical expectation is one 
possible way to check the accuracy of the explanation. However, this raises the 
question whether the explanation is faithful to the model. It could be possible 
that an attribution method gives a chemically correct explanation but may not be 
representative of the model. 


To check the faithfulness to the model of an attribution method, the substructure 
with the highest attribution can be removed to calculate the fidelity measure. 
Since,it is assumed that high attribution indicates an important substructure, 
removing that substructure should change the explanation is faithful to the model. 
For example, if the single hydroxyl group (ROH) is removed, the solubility will 
drop dramatically. Since the model is trained to predict the water solubility of 
small molecules without interaction with molecules other than water and itself, 
the graphical structure of the molecule must be taken into account. Generally, 
the scaffold (i.e. main skeleton of the molecule) will be the most hydrophobic 
(most negative attribution), but if removed, not one but possibly several substructures 
remain. As a result, the scaffold is disregarded to calculate fidelity.


When the most hydrophilic sbustructure is removed (referred to as positive fidelity), 
the it is expected that the predicted solubility decreases. Since the fidelity is 
defined as the difference between normal prediction and the perturbed prediction, it
is expected that the positive fidelity is a positive number. However, all attribution 
methods have a positive fidelity distribution around zero (\cref{fig:fidelity}). Negative
values of the positive fidelity are due to the presence of fully apolar molecules, where 
water solubility can increase when an apolar substructure is removed. Similarly, the negative 
fidelity (obtained by removing the most hydrophobic substructure) also has a distribution 
around zero. Overall, all attribution methods and prediction error groups have a similar, 
rather poor, fidelity.


It is also possible to compute the fidelity when removing the least important 
substructures, identified by an attribution close to zero (referenced by absolute 
negative fidelity). Then it is expected that the model prediction does not change 
much. This is also observed for all attribution methods, but the distribution has 
long tails (\cref{fig:absolute_fidelity}). Again, the different prediction error 
groups have a similar behavior.


\begin{figure}[h]
    \centering
    \includegraphics[scale=0.5]{fidelity.png}
    \caption{Distribution of fidelity obtained by removing the most positive 
        attribution (right side) or the most negative attribution (left side).
        Fidelity is computed for 100 molecules from the test data set.
    }
    \label{fig:fidelity}
\end{figure}


\begin{figure}[h]
    \centering
    \includegraphics[scale=0.5]{absolute_fidelity.png}
    \caption{Distribution of fidelity obtained by removing the most positive 
        attribution in absolute values (right side) or the most negative 
        attribution in absolute values (left side).
        Fidelity is computed for 100 molecules from the test data set.
    }
    \label{fig:absolute_fidelity}
\end{figure}
