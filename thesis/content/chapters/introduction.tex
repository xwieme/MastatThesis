\chapter{Introduction}


\section{Machine learning}

\section{Deep learning}


\section{Graph neural networks}


\section{Explainable machine learning}

\subsection{Shapley value}

Feature attribution methods in XAI assign a number to each feature implying how 
much that feature contributed to the model prediction.\cite{merrick2020explanation} 
In other words, the features cooperate with each other to obtain a payoff given 
by the ML model and the interest lies in the contribution of each feature to the 
model prediction. These problems are more generally studied in cooperative game 
theory. Formally, a cooperative game with transferable utility (i.e. a TU-game) is 
defined as a pair $(N, v)$ consisting of a set of players (i.e. the features) 
and a characteristic function (i.e. ML model) which satisfies\cite{zhang2022gstarx}


\begin{equation}
	v: 2^N \coloneqq \{S | S \subseteq N\} \rightarrow \mathbb{R}, \; v\left(\emptyset\right) = 0.
\end{equation}

A solution of a game $\phi(N, v) \in \mathbb{R}^{|N|}$ is a vector where the $i$th element 
denotes the contribution of player $i$ to the payoff $v(N)$ obtained by all players 
of the coalition $N$.\cite{zhang2022gstarx} Therefore, the solution vector, 
also called a value, can be used to provide the attributions of each feature. 

A popular value used in machine learning is the Shapley value, which is a unique 
solution that satisfies the following axioms:\cite{merrick2020explanation, shapley1953value}

\begin{itemize}

    \item Dummy player: If a player $i$ does not add to the payoff, then it receives a 
        zero value (i.e. $\forall S \subseteq N: v(S \cup \{i\}) = v(S) \implies \phi_i(N, v) = 0$).

    \item Symmetry: If two players ($i$ and $j$) have the same contribution in all coalitions, then 
        their values are equal (i.e. $\forall S \subseteq N \setminus \{i, j\}: v(S \cup \{i\}) = v(S \cup \{j\}) \implies \phi_i(N, v) = \phi_j(N, v)$).

    \item Efficiency: The sum of the attributions of all players equals the payoff of the coalition containing 
        all players (i.e. $\sum^{|N|}_i \phi_i(N, v) = v(N)$).

    \item Linearity: The value of a game where the characteristic function $v$ is a linear combination of 
        two other value functions $u$ and $w$, then the value is also a linear combination (i.e. 
        $v = \alpha u + \beta w \implies \phi(N, v) = \alpha \phi(N, u) + \beta \phi(N, w)$)

\end{itemize}

The Shapley value $\phi_i$ of player $i$ is given by the expected marginal contribution\cite{zhang2022gstarx}

\begin{equation}
	m(i) = v\left(S \cup \{i\}\right) - v\left(S\right), \; \text{with } S \subseteq N \setminus \{i\}
\end{equation}

over all possible coalitions

\begin{equation}
\begin{aligned}
	\label{eq:Shapley}
    \phi_i(N, v)  &= \frac{1}{|N|} \sum_{k=0}^{|N|-1} \frac{\left(|N| - 1 - k\right)k!}{\left(|N| - 1\right)!} \sum_{S \subseteq N \setminus \{i\}, |S| = k} m(i) \\
            &= \sum_{k=0}^{|N| - 1} \sum_{S \subseteq N \setminus \{i\}, |S| = k} \frac{\left(|N| - 1 - |S|\right)|S|!}{|N|!} m(i) \\
            &= \sum_{S \subseteq N \setminus \{i\}} \frac{\left(|N| - 1 - |S|\right)|S|!}{|N|!} m(i)
\end{aligned}
\end{equation}

assuming that every coalition is equally probable. This last assumption can be questioned whether it should 
always hold. For example, define an unanimity game with characteristic function 

\begin{equation}
    u_R = \begin{cases}
        1 \quad \text{if } R \subseteq S \\
        0 \quad \text{otherwise}
\end{cases}.
\end{equation}

Then the Shapley value for the unanimity game $(\{1, 2, 3\}, u_{\{1, 2, 3\}})$ is $1/3$ for every player.\cite{hamiache_value_1999} 
Now, suppose players one and three cannot communicate with each other and hence cannot form a coalition. Since 
the Shapley value does not take any communication structure into account, the Shapley values are still $1/3$ 
for all players. However, it would be more intuitive if player two has a higher value, as it can communicate 
to more players than players one and three. Because chemical structures can also be considered as graphs, where 
the atoms are nodes and bonds between atoms are edges, it could be interesting to take the limitation of communication 
into account. A value that is defined with a communication structure in minds is the Hamiache-Navarro value.\cite{hamiache_value_1999, hamiache_associated_2020}

\subsection{Hamiache-Navarro value}
