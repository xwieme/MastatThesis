\chapter{Introduction}


\section{Machine learning}

\section{Deep learning}


\section{Graph neural networks}


\section{Explainable machine learning}

\subsection{Shapley value}

Feature attribution methods in XAI assign a number to each feature implying how
much that feature contributed to the model prediction.\cite{merrick2020explanation}
In other words, the features cooperate with each other to obtain a payoff given
by the ML model and the interest lies in the contribution of each feature to the
model prediction. These problems are more generally studied in cooperative game
theory. Formally, a cooperative game with transferable utility (i.e. a TU-game) is
defined as a pair $(N, v)$ consisting of a set of players (i.e. the features)
and a characteristic function (i.e. ML model) which satisfies\cite{zhang2022gstarx}


\begin{equation}
	v: 2^N \coloneqq \{S | S \subseteq N\} \rightarrow \mathbb{R}, \quad v\left(\emptyset\right) = 0.
\end{equation}


A solution of a game $\phi(N, v) \in \mathbb{R}^{|N|}$ is a vector where the $i$th element
denotes the contribution of player $i$ to the payoff $v(N)$ obtained by all players
of the coalition $N$.\cite{zhang2022gstarx} Therefore, the solution vector,
also called a value, can be used to provide the attributions of each feature.


A popular value used in machine learning is the Shapley value, which distributes
the total payoff among the players in a mathematical fair manner by satisfying the
following axioms:\cite{merrick2020explanation, shapley1953value}


\begin{itemize}
	\item Dummy player: If a player $i$ does not add to the payoff, then it receives a
	      zero value (i.e. $\forall S \subseteq N: v(S \cup \{i\}) = v(S) \implies \phi_i(N, v) = 0$).

	\item Symmetry: If two players ($i$ and $j$) have the same contribution in all coalitions, then
	      their values are equal (i.e. $\forall S \subseteq N \setminus \{i, j\}: v(S \cup \{i\}) = v(S \cup \{j\}) \implies \phi_i(N, v) = \phi_j(N, v)$).

	\item Efficiency: The sum of the attributions of all players equals the payoff of the coalition containing
	      all players (i.e. $\sum^{|N|}_i \phi_i(N, v) = v(N)$).

	\item Linearity: The value of a game where the characteristic function $v$ is a linear combination of
	      two other value functions $u$ and $w$, then the value is also a linear combination (i.e.
	      $v = \alpha u + \beta w \implies \phi(N, v) = \alpha \phi(N, u) + \beta \phi(N, w)$)
\end{itemize}


The Shapley value $\phi_i$ of player $i$ is given by the expected marginal contribution\cite{zhang2022gstarx}


\begin{equation}
	m(i) = v\left(S \cup \{i\}\right) - v\left(S\right), \; \text{with } S \subseteq N \setminus \{i\}
\end{equation}


over all possible coalitions


\begin{equation}
	\begin{aligned}
		\label{eq:Shapley}
		\phi_i(N, v) & = \frac{1}{|N|} \sum_{k=0}^{|N|-1} \frac{\left(|N| - 1 - k\right)k!}{\left(|N| - 1\right)!} \sum_{S \subseteq N \setminus \{i\}, |S| = k} m(i) \\
		             & = \sum_{k=0}^{|N| - 1} \sum_{S \subseteq N \setminus \{i\}, |S| = k} \frac{\left(|N| - 1 - |S|\right)|S|!}{|N|!} m(i)                          \\
		             & = \sum_{S \subseteq N \setminus \{i\}} \frac{\left(|N| - 1 - |S|\right)|S|!}{|N|!} m(i)
	\end{aligned}
\end{equation}


assuming that every coalition is equally probable. This last assumption can be questioned whether it should
always hold. For instance, define an unanimity game as a game $(N, u_R)$ with characteristic function


\begin{equation}
	u_R = \begin{cases}
		1 \quad \text{if } R \subseteq S \\
		0 \quad \text{otherwise}
	\end{cases}.
\end{equation}


Then the Shapley value for the unanimity game $(\{1, 2, 3\}, u_{\{1,2,3\}})$ is $1/3$ for every player.\cite{hamiache_value_1999}
Now, suppose players one and three cannot communicate with each other and hence cannot form a coalition.
Since the Shapley value does not account for any communication structure, the Shapley values are still
$1/3$ for all players. However, it would be more intuitive if player two had a higher value, as it can
communicate to more players than players one and three. Because it is possible to represent chemical
structures as graphs, it could be interesting to include the graphical structure in the explanation. A
value that does include a communication structure is the Hamiache-Navarro value.\cite{hamiache_value_1999, hamiache_associated_2020}


\subsection{Hamiache-Navarro value}


Before discussing the Hamiache-Navarro value, some notation is defined. Let a game $(N, v)$ with a
communication structure be a triple $(N, v, g)$, where $g$ is a set of adjacent nodes\cite{hamiache_value_1999},


\begin{equation}
	g \subseteq g_N \coloneqq \{ \{i, j \} | i, j \in N \}.
\end{equation}

The graph $\left< N, g \right>$ specifies the communication structure, where players
$i$ and $j$ can only communicate if they are adjacent (i.e. $\{i, j \} \in g$).
A path is a sequence $i = i_1, i_2, \dots , i_k = j$ of nodes from player
$i$ to player $j$ such that


\begin{equation}
	\{i_q, i_{q+1} \} \in g, \; 1 \le q \le k - 1.
\end{equation}


If such a path exists between two players, then they are connected by the graph $\left<N, g\right>$
which is denoted by $i \underset{\left< N, g \right>}{\rightarrow} j$. This allows to
define a partition $N/g$ of the set of nodes $N$ over the graph $<N, g>$ as the set of all
connected nodes

\begin{equation}
	N/g = \{\{ i | i \in N \land i \underset{\left<N, g \right>}{\rightarrow} j\} \cup \{j\} | j \in N \}.
\end{equation}


A coalition $S \subseteq N$ can only directly interact with its connected neighbors. Let
$S^*$ be the set of players which are adjacent to at least one player in the coalition $S$.


\begin{equation}
	S^* \coloneqq \{ i \in N | \exists j \in S \text{ such that } \{i, j\} \in g \}.
\end{equation}

The interconnection of the neighbors of $S$ is of no importance as the coalition
$S$ cannot see them. Therefore, the coalition $S$ can value its own payoff $v^*(S)$ as its
payoff $v(S)$ in the original game and a fraction of the net payoff obtained by cooperating with
each of its neighbors separately\cite{hamiache2001associated}


\begin{equation}
	\label{eq:associated_game}
	v^*(S) =
	\begin{cases}
		\displaystyle
		v(S) + \tau \sum_{j \in S^* \setminus S} \left[ v(S \cup \{j\}) - v(S) - v(\{j\}) \right] & \text{if } |S/g| = 1 \\
		\displaystyle
		\sum_{R \in S/g} v^*(R)                                                                   & \text{otherwise}
	\end{cases}
	.
\end{equation}

This introduces a series of associated games $(N, v^*, g)$, $(N, v^{**}, g)$, \dots which converges
to the limit game $(N, \tilde{v}, g)$. An associated game is consistent if its solution is the same
as the solution of the original game. When the graph is complete (i.e. all players are connected to
each other), then the solution of the associated game converges to the Shapley value.\cite{hamiache2001associated}


In what follows, the concept of associated games are converted into a matrix approach.\cite{hamiache_associated_2020}
To be consistent, an order of the matrix columns and rows are defined. Therefore a lexicographic order is defined
for coalitions of the same size. Suppose $A$ and $B$ are coalitions of the same size where the elements are ordered
from small to large (i.e. $a_1 < a_2 < \dots < a_k, \quad a_i \in A$). The coalition $A$ comes lexicographic
before coalition $B$ ($A \prec B$) if and only if $a_1 < b_1$ or $\exists \gamma \in \mathbb{N}, 1 \le i < \gamma: a_i = b_i \land a_\gamma < b_\gamma$.
For two arbitrary coalitions $S$ and $T$, $S < T$ if $|S| < |T|$ or $S \prec T$. Using this ordering, the characteristic
function $v$ of the game $(N, v, g)$ can be represented as a vector in $\mathbb{R}^{2^N - 1}$. The associated game
\cref{eq:associated_game} $v^*_\tau = M_c v$ can than be written as a linear transformation of the characteristic vector
, where the square matrix $M_c$ for two arbitrary coalitions $\emptyset \ne S \subseteq N$
and  \emptyset \ne T \subseteq N$ is given by

	\begin{equation}
		M_c[S, T] =
		\begin{cases}
			1 - |N \setminus S| \tau & \text{if } |S| = |T|                         \\
			\tau                     & \text{if } |S| + 1 = |T| \land S \subseteq T \\
			-\tau                    & \text{if } |T| = 1  \land T \not \subseteq S \\
			0                        & \text{otherwise}
		\end{cases}
	\end{equation}


	and $P_g$,


\begin{equation}
	P_g[S, T] =
	\begin{cases}
		1 & \text{if } T \in S/g \\
		0 & \text{otherwise}     \\
	\end{cases}
\end{equation}
