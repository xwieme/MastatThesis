\chapter{Introduction}


\section{Machine learning}

The goal of machine learning is to construct a function $f$, based on parameters
$\pmb{\theta}$, that can accurately predict the outcome variable of the training
data and can generalize to new data.\cite{hastie2009elements} It is common to represent the function input
values (also known as features) as a matrix $\mathbf{X} \in \mathbb{R}^{N \times p}$ where
each row is one sample. Evaluation of the learning function $f$ using the feature matrix
results in the predicted values $\mathbf{\hat{Y}} = f(\mathbf{X}; \pmb{\theta})$. After
obtaining the predicted values, they are compared to the true values (also called labels)
in order to determine how good the model is, which introduces the concept of loss functions.
Different loss functions exists for different kind of problems.\cite{wang2020comprehensive}
A commonly used loss function is the mean squared error (MSE) (\cref{eq:mse})
which averages the squared difference between the predicted value and true label.
Furthermore, optimizing the loss function with respect to the learning function
parameters produces the best model with the given architecture.\cite{hastie2009elements}


\begin{equation}
	\label{eq:mse}
	\text{MSE} = \frac{1}{N} \sum_{i=1}^N \big[y_i - f(\mathbf{x}_i)\big]^2
\end{equation}


One of the simplest functions $f\left(\mathbf{X}; \pmb{\theta}\right) = \mathbf{X}\pmb{\theta}$
is a linear combination of the features, where $\pmb{\theta} \in \mathbb{R}^p$
is the vector of coefficients. This model, known as linear regression, is commonly
used in statistical modeling.\cite{kutner2005applied} Although the linear model is
very interpretable, the linearity constraint is too strict which limits its
applicability. Therefore, more complicated ML algorithms were developed which are
able to obtain improved performance on complicated data with respect to linear
regression by allowing non-linear relationships.\cite{deng2012mnist} However,
they pay a price in terms of interpretability.\cite{fan2021interpretability}


\section{Neural networks}


Neural networks are a class of machine learning algorithms which are able to
approximate any non-linear function.\cite{cybenko1989approximation} Initially, they where developed as
a simple mathematical model of the human brain.\cite{rosenblatt1962principles} Although this is a strong
simplification, the concept of neuron signal transit from axon to dendrites
is used as a basic building block for a neural network.\cite{wang2017origin} This signal transmission
takes place between units in a neural network. Then, different architectures
can be used to connect different units. The multi layer perceptron (MLP) is a simple
architecture that is frequently utilized in other, more complicated neural networks.\cite{almeida2020multilayer}
In an MLP three parts can be distinguished: an input layer, one or more hidden layers
and an output layer. The values of the input layer are equal to the given feature matrix
$\pmb{X}$. Taking $n$ different linear combinations of the input layer followed by
a non-linear activation function $\sigma$ results in the values of the first hidden
layer. Generally, the values $\pmb{A}^{(l)} \in \mathbb{R}^{N \times n^{(l)}}$ of layer
$l$ are given by


\begin{equation}
	\pmb{A}^{(l)} =  \sigma^{(l)} \left(\pmb{A}^{(l-1)}\pmb{\Theta}^{(l-1)} \right),
\end{equation}


where $\pmb{\Theta}^{(l-1)} \in \mathbb{R}^{n^{(l-1)} \times n^{(l)}}$ is the weights matrix
of layer $l - 1$, with $n^{(l-1)}$ and $n^{(l)}$ the number of units in layers
$l-1$ and $l$ respectively and $\pmb{A}^{(0)} = \pmb{X}$. Popular activation
functions are listed in \cref{tab:activation_functions}.

\begin{table}[h]
	\caption{Popular activation functions used in multi layer perceptrons.}
	\label{tab:activation_functions}
	\begin{center}
		\begin{tabular}{c|c}
			\toprule
			Sigmoid(x) = $ \frac{1}{1 + e^{-x}}$                 &
			Softmax(x) = $\frac{e^x}{\sum_{i=1}^{n_L} e^{x_i}}$                                                           \\
			\midrule
			ReLU(x)\cite{glorot2011deep} = $\begin{cases}
					                                x & \text{if } x > 0 \\
					                                0 & \text{otherwise}
				                                \end{cases}$        &
			LeakyReLU(x)\cite{maas2013rectifier} =  $\begin{cases}
					                                         x        & \text{if } x > 0 \\
					                                         \alpha x & \text{otherwise}
				                                         \end{cases}$
			\\ & with $\alpha \ge 0$
			\\
			\midrule
			SeLU(x)\cite{klambauer2017self} = $\lambda \begin{cases}
					                                           x               & \text{if } x > 0 \\
					                                           \alpha(e^x - 1) & \text{otherwise}
				                                           \end{cases}$
			                                                     &
			Gelu(x)\cite{hendrycks2016gaussian} = $\begin{cases}
					                                       x            & \text{if } x > 0 \\
					                                       x \, \Phi(x) & \text{otherwise}
				                                       \end{cases}$
			\\
			with $\gamma = 1.05070098$ and $\alpha = 1.67326324$ & with $\Phi(x)$ cumulative standard normal distribution \\
			\bottomrule
		\end{tabular}
	\end{center}
\end{table}

An MLP often does not assess the whole feature matrix at once. Rather, the feature
matrix is split up into many batches. Then, following each batch, the parameters are
updated using an optimization algorithm. Common optimization algorithms in neural networks
are gradient descent and Adam.


\subsection{Gradient descent}

\subsection{Adam}


\section{Graph neural networks}


\section{Explainable machine learning}


\subsection{Shapley value}
\label{subsec:shapley_value}

Feature attribution methods in XAI assign a number to each feature implying how
much that feature contributed to the model prediction.\cite{merrick2020explanation}
In other words, the features cooperate with each other to obtain a payoff given
by the ML model and the interest lies in the contribution of each feature to the
model prediction. These problems are more generally studied in cooperative game
theory. Formally, a cooperative game with transferable utility (i.e. a TU-game) is
defined as a pair $(N, v)$ consisting of a set of players (i.e. the features)
and a characteristic function (i.e. ML model) which satisfies\cite{zhang2022gstarx}


\begin{equation}
	v: 2^N \coloneqq \{S | S \subseteq N\} \rightarrow \mathbb{R}, \quad v\left(\emptyset\right) = 0.
\end{equation}


A solution of a game $\phi(N, v) \in \mathbb{R}^{|N|}$ is a vector where the $i$th element
denotes the contribution of player $i$ to the payoff $v(N)$ obtained by all players
of the coalition $N$.\cite{zhang2022gstarx} Therefore, the solution vector,
also called a value, can be used to provide the attributions of each feature.


A popular value used in machine learning is the Shapley value, which distributes
the total payoff among the players in a mathematical fair manner by satisfying the
following axioms:\cite{merrick2020explanation, shapley1953value}


\begin{itemize}
	\item Dummy player: If a player $i$ does not add to the payoff, then it receives a
	      zero value (i.e. $\forall S \subseteq N: v(S \cup \{i\}) = v(S) \implies \phi_i(N, v) = 0$).

	\item Symmetry: If two players ($i$ and $j$) have the same contribution in all coalitions, then
	      their values are equal (i.e. $\forall S \subseteq N \setminus \{i, j\}: v(S \cup \{i\}) = v(S \cup \{j\}) \implies \phi_i(N, v) = \phi_j(N, v)$).

	\item Efficiency: The sum of the attributions of all players equals the payoff of the coalition containing
	      all players (i.e. $\sum^{|N|}_i \phi_i(N, v) = v(N)$).

	\item Linearity: The value of a game where the characteristic function $v$ is a linear combination of
	      two other value functions $u$ and $w$, then the value is also a linear combination (i.e.
	      $v = \alpha u + \beta w \implies \phi(N, v) = \alpha \phi(N, u) + \beta \phi(N, w)$)
\end{itemize}


The Shapley value $\phi_i$ of player $i$ is given by the expected marginal contribution\cite{zhang2022gstarx}


\begin{equation}
	m(i) = v\left(S \cup \{i\}\right) - v\left(S\right), \; \text{with } S \subseteq N \setminus \{i\}
\end{equation}


over all possible coalitions


\begin{equation}
	\begin{aligned}
		\label{eq:Shapley}
		\phi_i(N, v) & = \frac{1}{|N|} \sum_{k=0}^{|N|-1} \frac{\left(|N| - 1 - k\right)k!}{\left(|N| - 1\right)!} \sum_{S \subseteq N \setminus \{i\}, |S| = k} m(i) \\
		             & = \sum_{k=0}^{|N| - 1} \sum_{S \subseteq N \setminus \{i\}, |S| = k} \frac{\left(|N| - 1 - |S|\right)|S|!}{|N|!} m(i)                          \\
		             & = \sum_{S \subseteq N \setminus \{i\}} \frac{\left(|N| - 1 - |S|\right)|S|!}{|N|!} m(i)
	\end{aligned}
\end{equation}


assuming that every coalition is equally probable. This last assumption can be questioned whether it should
always hold. For instance, define an unanimity game as a game $(N, u_R)$ with characteristic function


\begin{equation}
	u_R = \begin{cases}
		1 \quad \text{if } R \subseteq S \\
		0 \quad \text{otherwise}
	\end{cases}.
\end{equation}


Then the Shapley value for the unanimity game $(\{1, 2, 3\}, u_{\{1,2,3\}})$ is $1/3$ for every player.\cite{hamiache_value_1999}
Now, suppose players one and three cannot communicate with each other and hence cannot form a coalition.
Since the Shapley value does not account for any communication structure, the Shapley values are still
$1/3$ for all players. However, it would be more intuitive if player two had a higher value, as it can
communicate to more players than players one and three. Because it is possible to represent chemical
structures as graphs, it could be interesting to include the graphical structure in the explanation. A
value that does include a communication structure is the Hamiache-Navarro value.\cite{hamiache_value_1999, hamiache_associated_2020}


\subsection{Hamiache-Navarro value}


Before discussing the Hamiache-Navarro value, some notation is defined. Let a game $(N, v)$ with a
communication structure be a triple $(N, v, g)$, where $g \subseteq g_N \coloneqq \{ \{i, j \} | i, j \in N \}$
is a set of adjacent nodes. The graph $\left< N, g \right>$ specifies the
communication structure, where players $i$ and $j$ can only communicate if they are adjacent
(i.e. $\{i, j \} \in g$). A path is a sequence $i = i_1, i_2, \dots , i_k = j$ of nodes from player
$i$ to player $j$ such that $\{i_q, i_{q+1} \} \in g$ for $1 \le q \le k - 1$. If such a path exists
between two players, then they are connected by the graph $\left<N, g\right>$ which is denoted by
$i \underset{\left< N, g \right>}{\rightarrow} j$. This allows to define a partition
$N/g \coloneqq \{\{ i | i \in N \land i \underset{\left<N, g \right>}{\rightarrow} j\} \cup \{j\} | j \in N \}$
of the set of nodes $N$ over the graph $<N, g>$ as the set of all connected nodes.\cite{hamiache_value_1999}

A coalition $S \subseteq N$ can only directly interact with its connected neighbors. Let
$S^* \coloneqq \{ i \in N | \exists j \in S \text{ such that } \{i, j\} \in g \}$ be the set of players
which are adjacent to at least one player in the coalition $S$. The interconnection of the neighbors of
$S$ is of no importance as the coalition $S$ cannot see them. Therefore, the coalition $S$ can value its
own payoff $v^*(S)$ as its payoff $v(S)$ in the original game and a fraction $\tau$ of the net payoff obtained
by cooperating with each of its connected neighbors separately\cite{hamiache2001associated}


\begin{equation}
	\label{eq:associated_game}
	v^*(S) =
	\begin{cases}
		\displaystyle
		v(S) + \tau \sum_{j \in S^* \setminus S} \left[ v(S \cup \{j\}) - v(S) - v(\{j\}) \right] & \text{if } |S/g| = 1 \\
		\displaystyle
		\sum_{R \in S/g} v^*(R)                                                                   & \text{otherwise}
	\end{cases}
	.
\end{equation}

When the coalition $S$ is not connected (i.e. $|S/g| \ne 1$), then its own payoff is the sum of the self
evaluated payoffs of the components $R \in S/g$ of the coalition $S$.
This introduces a series of associated games $(N, v^*, g)$, $(N, v^{**}, g)$, \dots which converges
to the limit game $(N, \tilde{v}, g)$. An associated game is consistent if its solution is the same
as the solution of the original game. When the graph is complete (i.e. all players are connected to
each other), then the solution of the associated game converges to the Shapley value.\cite{hamiache2001associated}


In what follows, the concept of associated games are converted into a matrix approach.
To be consistent, an order of the matrix columns and rows are defined. Therefore a lexicographic order is defined
for coalitions of the same size. Suppose $A$ and $B$ are coalitions of the same size where the elements are ordered
from small to large (i.e. $a_1 < a_2 < \dots < a_k, \quad a_i \in A$). The coalition $A$ comes lexicographic
before coalition $B$ ($A \prec B$) if and only if $a_1 < b_1$ or $\exists \gamma \in \mathbb{N}, 1 \le i < \gamma: a_i = b_i \land a_\gamma < b_\gamma$.
For two arbitrary coalitions $S$ and $T$, $S < T$ if $|S| < |T|$ or $S \prec T$. Using this ordering, the characteristic
function $v$ of the game $(N, v, g)$ can be represented as a vector in $\mathbb{R}^{2^N - 1}$. The associated game
\cref{eq:associated_game} $v^*_\tau = P_g M_c P_g v$ can then be written as a linear transformation of the characteristic vector
, where the square matrix $M_c$ for two arbitrary coalitions $\emptyset \ne S \subseteq N$
and $\emptyset \ne T \subseteq N$ is given by\cite{hamiache_associated_2020,hamiache2010matrix}


\begin{equation}
	M_c[S, T] =
	\begin{cases}
		1 - |N \setminus S| \tau & \text{if } |S| = |T|                         \\
		\tau                     & \text{if } |S| + 1 = |T| \land S \subseteq T \\
		-\tau                    & \text{if } |T| = 1  \land T \not \subseteq S \\
		0                        & \text{otherwise}
	\end{cases}
\end{equation}


and matrix $P_g$ is given by

\begin{equation}
	P_g[S, T] =
	\begin{cases}
		1 & \text{if } T \in S/g \\
		0 & \text{otherwise}     \\
	\end{cases}
	,
\end{equation}


which represents the graphical structure of the game. This allows to write the series of associated
games as $v^*_\tau = P_g M_c P_g v, v^{**}_\tau = P_g M_c P_g v^*_\tau, \dots, v^{(k*)}_\tau = \left(P_g M_c P_g\right)^k v$.
Provided that $\tau$ is small enough ($\tau < \frac{2}{|N|}$ for complete graphs\cite{hamiache2001associated}),
the power series $\left(P_g M_c P_g\right)^k v$ converges as $k$ tends to infinity.\footnote{A proof can be found in \cite{hamiache_associated_2020}}
The convergence produces a limit game $(N, \tilde{v}, g)$, where the solution $\phi_i$ of player $i$ equals the
payoff $\tilde{v}(\{i\})$ of player $i$.\cite{zhang2022gstarx, hamiache_associated_2020}


\textbf{Numerical example}

In order to compare the Hamiache-Navarro value and the Shapley value the same example as in \cref{subsec:shapley_value}
is used. To recap, define an unanimity game with communication structure $(\{1, 2, 3\}, u_{\{1, 2, 3\}}, \{\{1, 2\}, \{2, 3\}\})$
where $u_{\{1, 2, 3\}}(S)$ is one if $\{1, 2, 3\} \in S$ and zero otherwise with $S \subseteq \{1, 2, 3\}$. As discussed
in \cref{subsec:shapley_value}, the Shapley value is equal to $1/3$ for all players. However, the HN-value is not equal
for all players. Players one and three have an HN-value of $1/4$ and the HN-value for player two is $1/2$
(see \cref{app:HN_example}).\cite{hamiache_associated_2020}
