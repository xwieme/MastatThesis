\chapter{Conclusion}


In this study, different attribution methods are compared within a chemical context. 
Based on the method developed by Wu et. al., a molecule is first split into chemically 
meaningful substructures. Subsequently the difference in model prediction and perturbed 
model prediction results in the importance of the removed substructure. Interactions between 
substructures are included by means of the Shapley value. Furthermore, the graphical structure 
is also incorporated in the explanation using the HN value. 


The HN value results in more narrow attribution distributions with respect to SME and Shapley values.
Also, the expected relation between functional groups is correctly captured by the HN value. However, 
all attribution methods provide chemically wrong explanations even when the prediction error is below 
the experimental error hinting at possible Clever Hans effect. 


Sometimes, the attribution methods do not provide the same explanation, even contradicting one another. 
SME and Shapley values provide for $78.07\%$ of the molecules the exact same substructure ranking, while 
Shapley and HN values agree for $74.26\%$ of the molecules. Deviation from a perfect Spearman rank correlation 
between the substructures is not associated with the prediction error. 


Agreement between attribution based substructure ranking and a chemically based ranking was explored.
A Friedman analysis revealed no statistically significant difference between the average Spearman 
rank correlcation of the three attribution methods. However, given that the p-values are near the 
nominal level and the current sample size is 100, a bigger sample size should be used in a follow up 
study to confirm these findings. Also a Wilcoxon-Mann-Whitney test between the different 
error groups is not statistically significant.


At last, the faithfulness of the explanation methods towards the model is quantified using different 
fidelity measures. Generally, all attribution are equally faithful to the model regardless of the prediction 
error with a rather poor fidelity.
