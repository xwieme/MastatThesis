%%%%%%%%%%%%%%%%%%%%%%%%%%%%%%%%%%%
%This is the LaTeX ARTICLE template for RSC journals
%Copyright The Royal Society of Chemistry 2016
%%%%%%%%%%%%%%%%%%%%%%%%%%%%%%%%%%%

\documentclass[twoside,twocolumn,9pt]{article}

\usepackage{tgheros}
\renewcommand{\familydefault}{\sfdefault}

\usepackage{extsizes}
\usepackage[super,sort&compress,comma]{natbib} 
\usepackage[version=3]{mhchem}
\usepackage[left=1.5cm, right=1.5cm, top=1.785cm, bottom=2.0cm]{geometry}
\usepackage{balance}
\usepackage{mathptmx}
\usepackage{sectsty}
\usepackage{graphicx} 
\usepackage{lastpage}
\usepackage[format=plain,justification=justified,singlelinecheck=false,font={stretch=1.125,small,sf},labelfont=bf,labelsep=space]{caption}
\usepackage{float}
\usepackage{fancyhdr}
\usepackage{fnpos}
\usepackage[english]{babel}
\addto{\captionsenglish}{%
  \renewcommand{\refname}{Notes and references}
}
\usepackage{array}
\usepackage{droidsans}
\usepackage{charter}
\usepackage[T1]{fontenc}
\usepackage[usenames,dvipsnames]{xcolor}
\usepackage{setspace}
\usepackage[compact]{titlesec}
%%%Please don't disable any packages in the preamble, as this may cause the template to display incorrectly.%%%


\usepackage{xcolor}
\definecolor{ugent_blue}{RGB}{30, 100, 200}
\definecolor{ugent_yellow}{cmyk}{.0, .10, 1, 0}

\usepackage{titlesec}
\titleformat{\section}
{\color{black}\normalfont\Large\bfseries}
{\color{black}\thesection}{1em}{}

\usepackage[colorlinks=true,linkcolor=black,citecolor=ugent_blue]{hyperref}

%\AtEveryCite{\color{ugent_blue}}




\usepackage{epstopdf}%This line makes .eps figures into .pdf - please comment out if not required.
\usepackage{physics}
\usepackage{cleveref}

\definecolor{cream}{RGB}{222,217,201}

\begin{document}

\pagestyle{fancy}
\thispagestyle{plain}
\fancypagestyle{plain}{
  %%%HEADER%%%
  \renewcommand{\headrulewidth}{0pt}
}
%%%END OF HEADER%%%

%%%PAGE SETUP - Please do not change any commands within this section%%%
\makeFNbottom
\makeatletter
\renewcommand\LARGE{\@setfontsize\LARGE{15pt}{17}}
\renewcommand\Large{\@setfontsize\Large{12pt}{14}}
\renewcommand\large{\@setfontsize\large{10pt}{12}}
\renewcommand\footnotesize{\@setfontsize\footnotesize{7pt}{10}}
\makeatother

\renewcommand{\thefootnote}{\fnsymbol{footnote}}
\renewcommand\footnoterule{\vspace*{1pt}% 
  \color{cream}\hrule width 3.5in height 0.4pt \color{black}\vspace*{5pt}}
\setcounter{secnumdepth}{5}

\makeatletter
\renewcommand\@biblabel[1]{#1}
\renewcommand\@makefntext[1]% 
{\noindent\makebox[0pt][r]{\@thefnmark\,}#1}
\makeatother
\renewcommand{\figurename}{\small{Fig.}~}
\sectionfont{\sffamily\Large}
\subsectionfont{\normalsize}
\subsubsectionfont{\bf}
\setstretch{1.125} %In particular, please do not alter this line.
\setlength{\skip\footins}{0.8cm}
\setlength{\footnotesep}{0.25cm}
\setlength{\jot}{10pt}
\titlespacing*{\section}{0pt}{4pt}{4pt}
\titlespacing*{\subsection}{0pt}{15pt}{1pt}
%%%END OF PAGE SETUP%%%

%%%FOOTER%%%
\fancyfoot{}
\fancyfoot[LO,RE]{\vspace{-7.1pt}\includegraphics[height=9pt]{head_foot/LF}}
\fancyfoot[CO]{\vspace{-7.1pt}\hspace{11.9cm}\includegraphics{head_foot/RF}}
\fancyfoot[CE]{\vspace{-7.2pt}\hspace{-13.2cm}\includegraphics{head_foot/RF}}
\fancyfoot[RO]{\footnotesize{\sffamily{1--\pageref{LastPage} {\color{ugent_yellow} ~\textbar } \hspace{2pt}\thepage}}}
\fancyfoot[LE]{\footnotesize{\sffamily{\thepage~{\color{ugent_yellow} ~\textbar }\hspace{4.65cm} 1--\pageref{LastPage}}}}
\fancyhead{}
\renewcommand{\headrulewidth}{0pt}
\renewcommand{\footrulewidth}{0pt}
\setlength{\arrayrulewidth}{1pt}
\setlength{\columnsep}{6.5mm}
\setlength\bibsep{1pt}
%%%END OF FOOTER%%%

%%%FIGURE SETUP - please do not change any commands within this section%%%
\makeatletter
\newlength{\figrulesep}
\setlength{\figrulesep}{0.5\textfloatsep}

\newcommand{\topfigrule}{\vspace*{-1pt}% 
  \noindent{\color{cream}\rule[-\figrulesep]{\columnwidth}{1.5pt}} }

\newcommand{\botfigrule}{\vspace*{-2pt}% 
  \noindent{\color{cream}\rule[\figrulesep]{\columnwidth}{1.5pt}} }

\newcommand{\dblfigrule}{\vspace*{-1pt}% 
  \noindent{\color{cream}\rule[-\figrulesep]{\textwidth}{1.5pt}} }

\makeatother
%%%END OF FIGURE SETUP%%%

%%%TITLE, AUTHORS AND ABSTRACT%%%
\twocolumn[
  \begin{@twocolumnfalse}
    {
      \includegraphics[width=18.5cm]{head_foot/header_bar}}\par
    \vspace{1em}
    \sffamily
    \begin{tabular}{m{4.5cm} p{13.5cm} }

                     & \noindent\LARGE{\textbf{Explainable graph neural networks $^\dag$}} \\%Article title goes here instead of the text "This is the title"
      \vspace{0.3cm} & \vspace{0.3cm}                                                                                                                                                 \\

                     & \noindent\large{Full Name,$^{\ast}$\textit{$^{a}$} Full Name,\textit{$^{b\ddag}$} and Full Name\textit{$^{a}$}}                                                \\%Author names go here instead of "Full name", etc.

                     &                                                                                                                                                                \\

                     & \noindent\normalsize{abstract}         \\%The abstrast goes here instead of the text "The abstract should be..."
    \end{tabular}

  \end{@twocolumnfalse} \vspace{1.6cm}

]
%%%END OF TITLE, AUTHORS AND ABSTRACT%%%

%%%FONT SETUP - please do not change any commands within this section
\renewcommand*\rmdefault{bch}\normalfont\upshape
\rmfamily
\section*{}
\vspace{-1cm}


% %%%FOOTNOTES%%%

\footnotetext{\textit{$^{b}$~Corresponding author:} \texttt{firstname.lastname@ugent.be}}

% %Please use \dag to cite the ESI in the main text of the article.
% %If you article does not have ESI please remove the the \dag symbol from the title and the footnotetext below.
% \footnotetext{\dag~Electronic Supplementary Information (ESI) available: [details of any supplementary information available should be included here]. See DOI: 00.0000/00000000.}
% %additional addresses can be cited as above using the lower-case letters, c, d, e... If all authors are from the same address, no letter is required

% \footnotetext{\ddag~Additional footnotes to the title and authors can be included \textit{e.g.}\ `Present address:' or `These authors contributed equally to this work' as above using the symbols: \ddag, \textsection, and \P. Please place the appropriate symbol next to the author's name and include a \texttt{\textbackslash footnotetext} entry in the the correct place in the list.}


%%%END OF FOOTNOTES%%%

%%%MAIN TEXT%%%%

\section{Introduction}
% Central ideas: Why did I do the work? What were the central motivations and hypotheses?

% The first paragraph or two should be written out completely. Pay particular attention to the opening sentence. Ideally, it should state concisely the objective of the work, and indicate why this objective is important.
% In general, the Introduction should have these elements: 

% \begin{itemize}
%     \item The objectives of the work. 
%     \item The justification for these objectives: Why is the work important? Cite the most important works \cite{whitesides2004a}.
%     \item Background: Who else has done what? How? What have we done previously?
%     \item Guidance to the reader: What should the reader watch for in the paper? What are the interesting high points? What strategy did we use?
%     \item Summary/conclusion: What should the reader expect as conclusion? In advanced versions of the outline, you should also include all the sections that will go in the Experimental section (at the level of paragraph subheadings) and indicate what information will go in the Microfilm section.
% \end{itemize}

Machine learning can be used to predict properties of molecules.\\cite{} As molecules can be 
represented by a graph (cf. Lewis structure\\cite{}), state of the art results are obtained 
using graph neural networks (GNN).\\cite{} However, the black box nature of GNN possesses a 
problem. If it is unknown how the model obtained a certain prediction, it could lead to 
distrust when the result will be used in performance critical applications or expensive 
practicall experiments like drugs discovery.\\cite{} 

The goal of explainable machine learning is to uncover the black box algorithms.\\cite{}
This will result in more trust in the model prediction and could also reveal
novel scientific knowledge of the studied application.\\cite{} Recently, Wu et. al used 
substructure mask exploration to explain molecular property models in a chemically 
intuitive manner.\cite{} Their approach uses the difference between the prediction 
of a molecule and the prediction when a functional group of that molecule is removed. 
An additional benefit of this method is that a linear model can be fitted to optimize 
molecular properties without the need of training an expensive machine learning model.

However, just using the difference could lead to wrong interpretations due to interaction 
effects.\cite{} A commonly used method to fairly distribute the prediction over the 
features is the Shapley value taken from cooperative game theory.\\cite{} This value cannot 
be applied in the context of GNN as the Shapley value does not take the graphical structure 
into account. Therefore the structure aware HM-value is used instead.\\cite{}

\section{Theory}

A game with transfer utility (TU game) is defined as a pair $(N, v)$ where $N$ is a finite 
set of players and $v$ is a characteristic function that maps for each coalition a pay-off

\begin{equation}

    v: 2^N := {S | S \subseteq N} \rightarrow \mathcal{R}.

\end{equation}

A game with communication structure is a triple $(N, v, g)$.
Let $g_N$ denote the set of all edges, i.e. 

$$
g_n  = \{\{i, j \} | i \ne j \and i, j \in N \}
$$

A graph is a pair $<N, g>$ where $g \subseteq g_N$ is the set of connected players.
Let $GR$ be the set of all graphs, i.e. 

$$
GR = \{ <N, g> | N \subseteq U \and g \subseteq g_n \}
$$

Let $<S, g(S)>$ denote the induced graph of $<N, g>$ with $S \subseteq N$ and $g(S) = \{\{i, j \} | i, j \in S}$.
A path between vertices $i$ and $j$ is a series of vertices $i_1 = i, i_2, \dots, i_k = j$ such that for all $q$ between 
$1$ and $k-1$ the edge $\{i_q, i_{q+1}\}$ is an element of $g$. If such a path exists, $i$ and $j$ are connected, which 
will be denoted as $i \underset{<N, g>}{\leftarrow}$.

Let $N/g$ be the partition of $N$ of all connected coalitions (called components) in $g$, i.e.
 
$$
    N/g = \{\{ i | i \underset{<N, g>}{\leftarrow} j \and i \in N \} \union \{j\} | j \in N \}.
$$

The set including its neighbors will be denoted by $S^* = \{ i \in N | \exitst j \in S such that \{i, j\} \in g} \cap S$.
The characteristic function $v/g$ is defined as the sum of the characteristic function $v$ over all components in $g$, 

$$
(v/g)(S) = \sum_{R \subseteq S} v(R).
$$

An unanimity game $(N, u_S)$ is defined as 

\begin{equation}
    u_S(T) = \begin{cases}

        1 & \text{if} S \subseteq T \\
        0 & \text{otherwise}

    \end{cases}

\end{equation}


\section{Methodology}

\section{Results and discussion}

\section{Conclusions}

%%%END OF MAIN TEXT%%%

%The \balance command can be used to balance the columns on the final page if desired. It should be placed anywhere within the first column of the last page.

\balance

%If notes are included in your references you can change the title from 'References' to 'Notes and references' using the following command:
%\renewcommand\refname{Notes and references}

%%%REFERENCES%%%
\bibliography{outline} %You need to replace "rsc" on this line with the name of your .bib file
\bibliographystyle{aip} %the AIP's .bst file

\newpage
\appendix
\markboth{Appendix}{}
\renewcommand{\thesection}{A.\arabic{section}}



\end{document}
