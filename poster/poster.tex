\documentclass[25pt, a0paper, portrait]{tikzposter}
\tikzposterlatexaffectionproofoff

\usetitlestyle{Empty}
\usebackgroundstyle{Empty}
\usepackage{wrapfig}
\usepackage[utf8]{inputenc}
\usepackage{physics}
\usepackage[mathscr]{euscript}  % \mathscr
\usepackage{mhchem}
\usepackage{subcaption}
\usepackage{mathtools}

\input{config/commands}
\bibliographystyle{unsrt}



%
%   TITLE
%

\title{
    \parbox{\linewidth}{ \center
        \HUGE{
            \textcolor{ugent_blue}{
                \textbf{
                    A chemical explanation of graph neural networks
                }
            }
        }
    }
}

\renewcommand\emph[1]{\textcolor{ugent_blue}{\textbf{#1}}}
\renewcommand\refname{\vskip -1cm}
\renewcommand{\familydefault}{\sfdefault}  % for the main font, use sans serif Computer Modern



%
%   AUTHORS AND AFFILIATIONS
%
\author[$\dagger$]{X. Wieme} 
\author[$\dagger$]{A. Gevaert} 
\author[$\dagger$]{Y. Saeys}


\affil[$\dagger$]{Ghent University, Krijgslaan 281 (S3), B-9000 Gent, België}

%
%   MAIN DOCUMENT
%

\begin{document}

\maketitle

\begin{columns}
    % Left Column.
    \column{0.5}
    % Introduction block.
    \block{Introduction} {
        % This block should always be there and should briefly introduce your research. Citations can be made as normal \cite{acke2020a}. If you want to \emph{emphasize} something, you can.

        \begin{itemize}

            \item \emph{Opaque predictions} for machine learning models builds \emph{trust issues} and restricts creation of 
            novel scientific insights found by those models.\cite{carvalho2019machine}.

        \item Current explainable artificial intelligence techniques \emph{fail} to provide a \emph{chemically intuitive 
            explanation}.\cite{yuan2022explainability, wu2023chemistry}

            \item Recently, Wu et. al. developed a chemically intuitive explanation technique for graph 
                neural networks by assigning attributions to \emph{chemical substructure} (e.g. functional groups, 
                BRICS, MURCKO).\cite{wu2023chemistry}

            \item The difference between the model prediction of a molecule and a masked molecule is a 
                too simple attribution. A popular method to distribute the model prediction over the 
                features is the \emph{Shapley value}.\cite{molnar2020interpretable} However, the Shapley value does not take the graphical 
                structure into account, which limits its use on \emph{graph neural networks}.\cite{zhang2022gstarx} 
                This restriction is solved by using the \emph{Hamiache-Navarro value}\cite{hamiache_value_1999}, which 
                extends the Shapley value to graphs and converges to the Shapley value on complete graphs.

        \end{itemize}

    }
    
    % Results block 2.
    \block{Research aims}{
        % This is where you can add the rest of your results.
        \begin{itemize}

            \item Explain graph neural networks using substructure mask exploration and the Hamiache-Navarro value.

        \end{itemize}

    }

    \column{0.5}
    % Theory
    \block{Theory}{
        % Add a small theory section. Keep your theory short and to the point.
        \section{Graph neural networks}

        \section{Shapley value}

        Let $(N, v)$ be a game with a set of players $N$ and a characteristic function 
        satisfying $2^N \coloneqq \{S | S \subseteq N\} \rightarrow \mathcal{R}, \; v(\emptyset) = 0$.
        The Shapley value $\phi_i$ of player $i$ is defined as\cite{shapley1953value}

        \begin{equation}
            \phi_i = \sum_{S \subseteq N \ \{i\}} \frac{\left( |N| - 1 - |S| \right)! |S|!}{|N|!} \left(v(S \cup \{i\}) - v(S) \right)
        \end{equation}

        \section{Hamiache-Navarro value}

        Let $(N, v, g)$ be a game with communication structure given by the graph $\left< N, g \right>$, where 
        $g \subseteq g_N \coloneqq \{ \{i, j\} | i, j \in N \}$ is the set of adjacent players. The 
        associated game $(N, v^*, g)$ is defined as

        \begin{equation}
            v^*(S) = 
            \begin{cases}
                v(S) + \tau \sum_{j \in \mathcal{N}(S)} \left[ v(S \cup \{j\}) - v(S) - v(\{j\}) \right]  & \text{if } |S/g| = 1 \\
                \sum_{R \in S/g} v^*(R) &  \text{otherwise}
            \end{cases}
            .
        \end{equation}
    }
\end{columns}

    % Results block 1.
    \block{Hamiache-Navarro value results in narrower distribution with respect to the difference}{
        % Add an initial part of your results here
        
        \begin{tikzfigure}
            \includegraphics[scale=1.5]{../data/images/esol_attribution_distributions.png}
        \end{tikzfigure}

    }

\begin{columns}
    \column{0.5}
    % Block containing the references.
    \block{References} {
        \vspace{-1cm}
        \bibliography{poster.bib}
    }

    % Right Column.
    \column{0.5}

    \block{substructure mask explanation}{

        \hspace{10cm} \textbf{Difference}\cite{wu2023chemistry} \hspace{13cm} \textbf{HN}
        \begin{tikzfigure}
            \includegraphics[scale=1.5]{../data/images/dimethyl_phthalate_difference.png}
            \hspace{2cm}
            \includegraphics[scale=1.5]{../data/images/dimethyl_phthalate_HN.png}
        \end{tikzfigure}
    }


    % Contact and Acknowledgements block.
    \block{Contact and acknowledgements} {
        Contact via email at xander.wieme@ugent.be. \\
        \begin{wrapfigure}[1]{r}{10cm}
            \vspace{-4cm}
            \begin{tikzfigure}[]
                \includegraphics[height=4cm]{figures/ugent_logo}
            \end{tikzfigure}
        \end{wrapfigure}
    }


\end{columns}

\end{document}
